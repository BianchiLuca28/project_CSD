\chapter{Introduzione}
Questa tesi presenta alcuni aspetti teorici e pratici della generazione di numeri pseudocasuali. 
In particolare, si pone l'attenzione sulla $\textit{Teoria del caos}$ e su alcuni algoritmi di generazione, i cui risultati sono stati testati tramite l'utilizzo di uno standard (NIST SP-800-22), formato da 15 test, il cui focus principale è quello di testare la validità crittografica di una sequenza pseudocasuale.

Le motivazioni che mi hanno spinto ad approfondire tale tema sono di duplice natura.
Innanzitutto, l'interesse per l'argomento è nato in aula, quando si è detto che la generazione di numeri effettivamente random è un problema irrisolto nella matematica.
Dopodiché, la curiosità mi ha spinto ad informarmi sull'argomento, arrivando così ad apprezzare sia la matematica presente dietro le quinte di un generatore pseudocasuale, sia le applicazioni pratiche: essendo io un programmatore, ho apprezzato particolarmente quelle crittografiche.

L'obiettivo di questa tesi di laurea è quello di approfondire tutto il processo di creazione di un generatore pseudocasuale. 
Si parte dalle proprietà matematiche che ci permettono di definire un generatore come tale, per poi svilupparne diversi tipi con l'obiettivo ultimo di testarne l'affidabilità. Grazie al lavoro svolto, è stato possibile studiare i comportamenti di diversi generatori in un contesto crittografico, nel quale l'affidabilità delle sequenze casuali deve essere assolutamente garantita.

La tesi è articolata in tre capitoli: 

Nel Capitolo \ref{chapter:teoriaCaos} si introducono i $\textit{sistemi dinamici}$, essenziali per definire quello che è il focus principale di questo capitolo: il $\textit{caos}$.
Per definire il caos è stato utilizzato un approccio topologico.
Il capitolo, infine, si conclude con la definizione di due sistemi dinamici, che saranno utilizzati poi per la creazione di due PRNG (Pseudo Random Numbers Generator): la $\textit{Mappa Logistica}$ ed il $\textit{Sistema di Lorenz Generalizzato}$, rispettivamente, un sistema dinamico discreto ed uno continuo.

Nel Capitolo \ref{chapter:algoGen} si applica la teoria del Capitolo \ref{chapter:teoriaCaos} per creare due generatori basati sui sistemi dinamici appena nominati, mappa logistica e sistema di Lorenz generalizzato.
Come ulteriore esercizio pratico, viene introdotto l'algoritmo di Box-Muller, che permette di generare una sequenza di numeri con distribuzione gaussiana, partendo dalle sequenze uniformemente distribuite ottenute tramite i due PRNG precedenti. 
Chiaramente, non essendo quest'ultima uniformemente distribuita, non è stato possibile applicare gli stessi test utilizzati per le altre due sequenze.
In particolare, ne vengono testate la curtosi, la simmetria e la normalità. 


Nel Capitolo \ref{chapter:chapter4} si testano i vari algoritmi.
Si parte con una definizione rigorosa di $\textit{variabili casuali}$ e $\textit{random walk}$, due oggetti che serviranno poi a comprendere i test.
Le definizioni continuano anche nella sezione successiva, in cui viene spiegato a cosa ci si riferisce quando in un test sull'ipotesi statistica si parla di ipotesi nulla, di ipotesi alternativa e di errori di tipo I e II.
Il capitolo prosegue con una spiegazione dettagliata dei 15 test utilizzati per validare i PRNG sviluppati.
Infine, si trovano le applicazioni di tali test alle sequenze generate con i due generatori sviluppati e, per completezza, anche su due funzioni fornite dal linguaggio di programmazione C++, con le varie conclusioni per ognuno dei generatori testati.
